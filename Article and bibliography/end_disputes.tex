\documentclass{article}

% these lines make double-spaced with wide margins for submission purposes
% comment them out to make it look like a more conventionally formatted article
\usepackage[margin=1in]{geometry}
\usepackage{setspace}
\AtBeginDocument{\doublespacing} % activates double spacing for the entire document, footnotes included

\usepackage[
backend=bibtex]{biblatex}
\usepackage{graphicx}
\usepackage [english]{babel}
\usepackage [autostyle, english = american]{csquotes}
\MakeOuterQuote{"}
\usepackage{amsmath}
\DeclareMathOperator*{\argmax}{arg\,max}
\DeclareMathOperator*{\argmin}{arg\,min}
\usepackage{BOONDOX-cal} % a calligraphic font that includes lowercase letters, will be used with mathcal command
\usepackage{babel, blindtext}
\usepackage{caption}
\usepackage{hyperref}
\usepackage{amssymb}
\newenvironment{nohyphen}
  {\tolerance=1% Also consider setting \pretolerance
   \emergencystretch=\maxdimen%
   \hyphenpenalty=10000%
   \hbadness=10000}% \begin{nohyphen}
  {\par}% \end{nohyphen}
 
\addbibresource{end_disputes.bib}

\begin{document}

%\title{A Cradle-to-Grave \\ Algorithmic Game Theory \\ Model of the Tort System}
\title{Evaluating Procedural Devices \\ in an Equilibrating Civil Litigation System}
\author{Michael Abramowicz \\ \href{mailto:abramowicz@law.gwu.edu}{abramowicz@law.gwu.edu} \\ George Washington University Law School}

\maketitle

\begin{abstract}
\begin{nohyphen}
[WRITE ABSTRACT]
\end{nohyphen}
\end{abstract}

\section{Introduction}

In a world of costly litigation, how can procedural mechanisms best advance the cause of the substantive law? This question lies at the heart of much of the law-and-economics literature on issues such as fee shifting, damages multipliers, and burdens of persuasion. Answers, however, have proven elusive. Models suggest that a wide range of policy might be optimal in the real world. Rowe (1984) concludes "that no general prediction of the relative overall effects of the American and English rules ... seems possible." Polinsky and Rubinfeld (1998), meanwhile, shows that changing assumptions about the settlement process could reverse what had been the general result that the English rule better discourages plaintiffs with low probabilities of prevailing from going to trial. With regard to damages multipliers, the canonical theory presented by Polinsky and Shavell (1998) is that total damages should equal harm divided by the probability of detection. Yet Craswell  (1999) shows that constant multipliers may be more administrable while still providing optimal deterrence. And studying burdens of proof, Kaplow (2012) points out how adding consideration of an additional policy dimension could reverse the conclusions of prior models. 

Part of the challenge for the theoretical literature is that any mathematical model can incorporate only so many features of the litigation process before the analysis becomes intractable. Even the sign of predicted effects may change depending on which features are included, and even a single model may produce different recommendations, depending on the value of a particular parameter. Models designed to address one procedural choice, such as burdens of proof, may not be well suited to addressing others. The theoretical literature on procedural devices is thus highly fractured. It offers nuanced insights into how particular assumptions may affect the case for different interventions into the civil litigation process, but researchers have not converted on generally agreed upon conclusions about optimal system design. And such convergence seems impossible unless any given theoretical excursion can improve the realism of modeling of settlement bargaining, consider multiple procedural devices in tandem, and more explicitly consider the implications of trade-offs for social welfare.

An understandable reaction may be a flight from theory. But Greiner and Matthews (2016), in their comprehensive review of randomized controlled trials in law, find no tests of devices like fee shifting, damages multipliers, or changing burdens of proof in civil trials. Even if a legislature were to approve randomizing litigants into these regimes, such experiments would not be able to measure effects on primary behavior, unless litigants were randomized before cases even accrued. Natural experiments in theory could reveal effects on primary behavior variables such as accident rates. But natural experiments with strong identifications are few and far between; Helmers et al. (2021) reports a valuable experiment on fee-shifting but compares only two venues. Experimental evidence leaves questions about external validity. And so we can hve little confidence that we have arrived at the optimal approaches to these most foundational aspects of our litigation system. It should not be surprising that procedural design choices vary across countries (the American vs. the British rules are so named for a reason) and legal contexts (with antitrust and the False Claims Act in the United States being exceptional in imposing fixed treble damages and with heightened burdens of proof applying in many contexts).

The prospect that any paper with any methodology will get us to the correct answer thus seems small. But we may be able to make some progress by using new tools, particularly taking advantage of the increasing power of computation. This paper seeks to use algorithmic game theory to develop a single theoretical model that can be applied to questions of fee shifting, damages multipliers, and burdens of proof, while offering more realism than one-sided information models. The expectation is not that we will produce definitive answers about optimal design. But by varying procedural rules and other assumptions with an approach that allows more knobs and variations than most models, perhaps we might at least be able to develop a sense of the answer to the following important meta-question: How much does the choice of procedural mechanism really matter? 

Answering this question, however, requires more than a model of the litigation process itself, even with a better characterization of that process than before. To address implications of design for overall welfare, we cannot look at litigation outcome variables--such as settlement rates or accuracy--in isolation. Rather, we must assess the ramifications of changes in the litigation system for primary behavior. As Shavell (1997) recognized, litigation is rife with externalities. The mere expectation of litigation may affect third parties through deterrence (and perhaps overdeterrence), and litigants impose externalities on one another in litigation. There can thus be too much litigation or too little, but procedural mechanisms can affect the calibration of such incentives. To fully tally the consequence of different choices of procedural mechanisms, we must examine not only the effects of such mechanisms on any given suit, but also on the legal system as a whole. Even if we cannot be sure of the precise consequences of procedural choices for litigation costs, we may be able to gain appreciation of the extent to which the system can equilibrate. For example, to what extent might increases in the cost of trying any particular case be made up by a reduction in the number of cases overall, either because fewer disputes arise (perhaps because of less injury) or because more cases settle?

Given the challenges of mathematical modeling, the theoretical literature has for the most part avoided the challenge of simultaneously modeling both the generation of disputes and the litigation process that resolves them. Models of the tort process often take the litigation system as a given, for example by assuming a fixed cost of litigation cost without fully assessing the dynamics of trial. Shavell (1980)'s canonical comparison of strict liability versus negligence assumes the absence of litigation costs, and even many more modern models, such as Baumann and Rasch (2024)'s model of product liability, reasonably adopt a similar assumption. Meanwhile, models of the litigation process itself, such as models of fee shifting, may begin from the assumption that a dispute has occurred, thus ignoring the possibility that a different litigation system might have avoided an accident or other source of controversy. Argenton and Wang (2023) is just one of many recent examples that assume the existence of a dispute because doing so allows undistracted focus on the litigation process.

Some modelers have commented on the design of the litigation process using models in which the generation of disputes is endogenous, but have been forced to accept great simplifications about the process of adjudication itself. Png (1987)'s model is one of the first in which injurers vary in the cost of care and choose care endogenously, and the entire game is simplified into a extensive form game tree with only 15 nodes. His model demonstrates that the litigation process shapes incentives for care but leaves open many interesting questions of institutional design. Polinsky and Rubinfeld (1988) show that optimal liability may require an increase or decrease in damages to balance the goals of optimizing care and minimizing litigation expenditures, but they unrealistically assume no judicial error. Spier (1994) provides a sophisticated model of settlement bargaining and explores trade-offs with respect to whether damages awards should be finely tuned to the level of harm, but she too assumes no errors in measuring liability. Meanwhile, Polinsky and Shavell (2014) allow for litigants to face fixed and variable costs of litigation depending on the level of damages, but courts again adjudicate without error. Yet the possibility of judicial error may be critical to assessing the efficiency of procedural mechanisms. If, for example, treble damages are assessed sometimes against a not truly liable party, that might be problematic even if the treble damages help mitigate the danger of underdeterrence. 

One scholar has furnished a model with endogenous disputes and imperfect adjudication, Keith Hylton. Over three papers, Hylton develops what he calls a "cradle to grave" model of litigation. In his models, potential injurers must decide how much to take care, and these decisions determine how many accidents occur, which in turn result in lawsuits in which there might or might not be fee shifting. In Hylton (1990), Hylton (1993) and Hylton (2002), the potential tortfeasor draws a random cost of care and then compares it to the benefit in expected liability savings from compliance, taking into account the risk of type-1 and type-2 errors. The first article does not explicitly model settlement, while the second assumes that cases settle whenever there is a zone of agreement (that is, the plaintiff's minimum acceptable settlement is not greater than the defendant's maximum offer). The third article features a Bayesian settlement model in which plaintiffs play mixed strategies in determining whether to accept defendants' offers.  

To accomplish the remarkable feat of finding equilibria across both care and litigation decisions, Hylton necessarily must make some sacrifices in the realism of the model of litigation. While these models admit of judicial error, the risk of error is no greater when the level of care is close to the legal standard than when it is far from it. In addition, the strategic interactions between plaintiffs and defendants are stylized in order to yield tractable equilibria: bargaining is reduced to simple settlement zones in the second paper, and the informational structure is entirely one sided in the third of the trilogy, thus avoiding the methodological complexity of two-sided asymmetric information. In reality, though defendants may have more knowledge than plaintiffs of their own care decisions, that advantage is mitigated by discovery. Thus, parties are likely both informed about the probability of liability to some imperfect degree, with differences in 

Recent literature, meanwhile, has made progress in modeling two-sided asymmetric information, though without Hylton's innovation of endogenous disputes. Earlier models, such as Bebchuk (1984) and Daughety and Reinganum (1994), allowed for two-sided asymmetric information by granting each party information on different quantities, for example with one party knowing the level of damages and the other party knowing the level of liability. But Friedman and Wittman (2007) and Klerman, Lee and Liu (2018) model the situation in which each party has independent private information about the same issue, such as the level of damages. Dari-Mattiacci and Saraceno (2020) manage to incorporate an analysis of the effect of fee shifting on such information. Following Friedman and Wittman, Dari-Mattiacci and Saraceno assume that each party receives an independent signal of the level of damages and that the total damages is equal to the sum of the signals.

As Abramowicz (2025a) points out, however, tractability required a number of critical assumptions, such as relatively low litigation costs, that every potential lawsuit is always contested, and that a linkage exists between the degree of information asymmetry and the true merits of the case. Rather than complicate an already intricate mathematical model, that article switches to using a computational game theory algorithm to study similar models in which the outcome of litigation depends on the sum of the parties' signals. The linear programming algorithm, developed by von Stengel, van den Elzen, and Talman (2002), identifies exact perfect Bayesian Nash equilibria in extensive form game trees. Abramowicz (2025a) applies this algorithm to extensive form games of over 16,000 nodes, allowing incorporation and variation of a number of game features that could not easily be jointly modeled mathematically. These features include uncertainty about liability rather than damages, risk aversion, and options not to sue or defend, both at the outset of litigation and after the failure of negotiation.

The results strongly suggest that generalizations based on simpler models may not survive more robust specification. For example, while Dari-Mattiacci and Saraceno found that the English (loser pays) rule would be relatively advantageous in a jurisdiction with relatively low litigation costs and the American rule would be advantageous in a jurisdiction with relatively high litigation costs. None of the major computational specifications supported both of these results. Of course, computational results will not necessarily be robust either. Abramowicz (2025b) modifies the earlier model, featuring game trees of over 45,000 nodes that enable correlated instead of independent signals. That is, each party receives a noisy estimate of the true value of the litigation. In this setting, fee shifting can have complicated strategic effects. On one hand, it may lead parties with weak cases to give up or settle on unfavorable terms. On the other, especially when costs are relatively high, it may encourage a party to bluff in the hope that the other party will quit first. 

A significant limitation of these computational analyses is that they feature disputes that are exogenously generated, with a $\frac{1}{2}$ chance of a case being one in which the defendant is truly liable and an equal chance of being one in which the defendant is not truly liable. Truly liable cases are assumed tend to have stronger evidence for the plaintiff than not truly liable cases, but the distribution of litigation strength given true liability is also set based on an arbitrary parameter. In essence, in these models, as in many litigation models, cases in effect fall from the sky. This leaves unresolved questions of whether the conclusions would be different if there were a different distribution of truly liable vs. not truly liable cases. More critically, it elides the interaction effects between the litigation process and the generation of disputes. The computational approach to date has modeled how fee-shifting affects cost and accuracy in various environments, but it has not previously combined these considerations, as Hylton does, in an effort to identify global optimality.

This article applies algorithmic game theory, albeit with a different algorithm that makes it feasible to feature much larger game trees, each containing over 225,000 nodes. The greater size of these game trees makes it feasible to integrate potential tortfeasors' decisions about whether to take care, while still maintaining the richness of the bargaining framework of the earlier papers. It thus continues Hylton's project of modeling litigation cradle to grave, while allowing for two-sided asymmetric information. Each party's signals span multiple levels of litigation quality, and settlements can occur at each of multiple payment levels. Instead of one party being limited to accepting or rejecting the other party's settlement offer, the parties follow the Chaterjee and Samuelson (1983) bargaining protocol, in which each party submits an offer and cases settle at the midpoint if the defendant's offer equals or exceeds the plaintiff's. Although the algorithm uses floating point numbers and finds approximate equilibria, unlike the exact algorithm operating on rational numbers of the von Stengel et al. algorithm, these equilibria are measurably quite close to perfect Bayesian Nash equilibria. 

This article's analysis is not limited to fee shifting. Recognizing Craswell (1999)'s point that a variety of different types of procedural mechanisms might help achieve optimality, this article also considers two other procedural variations: damages multiples and alterations in the liability threshold. Some simulations intersect multiple such variations, to enable consideration of, for example, a regime in which it is difficult to establish liability but liability once established is heavy. With each simulation, it is possible to observe the sum of precaution, injury, and litigation costs, and differences across simulations can thus be at least tentatively attributed to differences in parameters and, in particular, differences in the procedural mechanisms used. Though it is difficult to encapsulate all of the article's findings in a single conclusion, a central point emerges: When lawsuits are generated endogenously based on the potential defendant's level of care, the adjustments in that level to the legal environment will generally prevent extreme variation in total costs. If liability levels are high, for example, precaution will be relatively high, but injury and litigation will be relatively low. That does not mean that the procedural mechanisms are entirely irrelevant, and certainly at extremes of the cost continuum (very low or very high litigation costs), some clear conclusions emerge. But the effects of even aggressive procedural mechanisms such as damages multiples are much more muted than one might think. 

[SUMMARY OF CONCLUSIONS]

\section{An Extensive Form Endogenous Litigation Game}

The approach that would be most consistent with Hylton's and indeed the broader literature on the economics of accidents would be to allow the potential defendant to choose a level of care, aware of the implications of that level for the probability that an accident occurs. Though perhaps plausible in some cases, this approach would have a serious drawback: When a case arises, the defendant would necessarily know how much care was exerted and therefore the quality of the lawsuit. The model would thus become a one-sided asymmetric information model. Yet in reality, both plaintiffs and defendants will often be uncertain about the outcome of litigation. Moreover, after initial accumulation and sharing of information, whether through formal discovery or otherwise, it is not obvious that the defendant's knowledge in every case will necessarily be better than the plaintiff's. Such an asymmetry might manifest in many cases, but it would be preferable if that possibility could be explored as one setting of the model. A neutral default setting would be one in which each side has equal insight into litigation quality. 

A modeling approach addressing this concern emerges from the recognition that in many real settings, both parties can observe what level of precaution the injurer actually took (for example, whether a machine guard was installed or which safety protocol was followed), and may even have strong information about the cost of such precautions. Meanwhile, the parties also may know what precautions the injurer forewent and how much such additional precaution might have cost. Nonetheless, there might still be uncertainty concerning how \emph{effective} the marginal precaution not taken would have been at reducing risk and whether such a risk reduction would have been cost-justified. Effectiveness depends on technological fit, local conditions, and other hidden factors that are not fully verifiable or knowable ex ante or perhaps even after an accident occurs. Such questions can be particularly challenging because they are counter-factual. Disputes in tort cases will thus often concern whether the defendant should have taken a more cautious path than the defendant actually took. To be sure, there may sometimes be debates as well as to what the defendant actually did and as to what a hypothetical precaution would have cost, but even with computation, analytic tractability makes desirable focus on a single source of uncertainty. For the purpose of the model in this article, the fount of uncertainty will thus concern the level of \emph{precaution power}, that is the extent to which expenditures on precaution would have reduced the probability of an accident. Each party knows the \emph{precaution level} but receives only a signal of precaution power. 

This section thus presents a single extensive-form model in which precaution and litigation are determined jointly in equilibrium under a negligence rule with imperfect adjudication and two-sided private information. Chance determines an underlying state that governs how effective precaution is at reducing accident risk. The defendant chooses a discrete precaution level before any accident occurs. The precaution level and chance then jointly determine whether the defendant causes an accident. If the defendant does not cause an accident, there is nonetheless some chance, unrelated to the level of precaution, that an accident is wrongly attributed to the defendant. 

Assuming an accident occurs, the plaintiff decides whether to file suit, the defendant decides whether to answer or default, the parties engage in a one-shot simultaneous-offers bargaining stage (with acceptance if the defendant’s offer weakly exceeds the plaintiff’s demand). If there is no settlement the case proceeds to trial unless one party decides to quit upon settlement failure. At trial, a court renders a binary liability verdict based on the liability threshold and on its own noisy signal of precaution power, again assuming knowledge of the precaution level. Payoffs incorporate damages (with an optional multiplier), litigation costs, and fee rules. Social welfare accounts for precaution costs, defendant-caused accident losses, and litigation costs. 

In this section, we will introduce the game by zooming into portions of a miniature version of the extension form game tree and then discussing the algorithm used to find perfect Bayesian Nash equilibria in these trees. More detail about the construction of the game trees is available in the appendices. Appendix A elaborates the formulas that are used for critical calculations, such as the probability of an accident and the probability that the court rules for the plaintiff in a case that goes to trial. Appendix B, meanwhile, describes a simplification that makes calculation much more feasible. The simplification is that the node in which the precaution power level is chosen can be eliminated from the game tree, so long as Bayesian probability is applied both to calculations (such as accident and trial probabilities) and to back out signal values after games are complete. This simplification was tested in relatively small trees by comparing its results to the same simulations run without simplification, and they produced identical results. The details underlying this algorithmic shortcut may be of interest primarily to readers interested in implementing similar simulations.

\subsection{Extensive-form game tree}

An extensive-form game tree can fully capture the formal game summarized above for any particular set of parameter values. The game features two strategic players and Chance. The \emph{defendant} (potential injurer) chooses whether to engage in the underlying activity, and, if so, a discrete precaution level. (The choice whether to engage in the activity and thus receive the fruits of that activity has little role in the analysis in this paper, but is included both for analytic completeness and as a source for analysis in future work.) The \emph{plaintiff} (victim) decides whether to file, whether to quit before trial if settlement fails, and which demand to submit in simultaneous offers. The defendant similarly decides whether to answer, whether to quit before trial if settlement fails, and what demand to submit. The \emph{court} is modeled as a chance node that observes a noisy liability signal at trial and returns a binary liability verdict.

In the baseline calibration used for analysis, the number of hidden precaution-power states is $n_H=8$, the number of discrete precaution levels is $n_K=8$, and each party’s liability signal has $n_{S_P}=n_{S_D}=8$ levels. In addition, the number of possible settlement offers that the plaintiff and defendant, respectively, may make, is defined as $n_{\mathcal{o}_P} = n_{\mathcal{o}_D} = 8$. In the miniaturized version of the game tree explored here, $n_H = n_K = n_{S_P} = n_{S_D} = n_{\mathcal{o}_P} = n_{\mathcal{o}_D} = 2$ (for the hidden state, precaution levels, and the plaintiff/defendant signals and offers). Otherwise, parameters are set to the baseline values that we use in the simulation with the larger tree. Figure \ref{fig:smalltree.pdf} provides a zoomed-out view of the miniaturized version of the game. This is designed solely to provide context and a sense of the game complexity even in this miniature game. We must now zoom into particular portions of the game tree to allow for clearer explication of the game.

 \begin{figure}[t]
  \centering
  \includegraphics[width=0.8\textwidth,height=0.3\textheight]{../Figures/smalltree.pdf}
  \caption{Zoomed out version of the miniaturized version of the game tree.}
  \label{fig:smalltree.pdf}
\end{figure}

The game proceeds in three loosely stages. For each stage, we first present the technical structure of the game, and then discuss how the corresponding figure illustrates that structure. 

Figure \ref{fig:smalltree_beginning.pdf} provides the first stage in the miniature version of the game. The diagram shows the split of hidden states representing precaution power at the left, the branching for defendant and plaintiff signals just to the right, the defendant’s decision to engage in the activity and select a precaution level, and the accident node at the far right. Ellipses are used to indicate the vast portions of the miniaturized game tree that are omitted. 

The numbers attached to these nodes, along with the probability estimates from the calculated Nash equilibrium in the miniaturized game, provide some intuition. In the portion of the tree fully presented in Figure \ref{fig:smalltree_beginning.pdf}, the precaution power is set randomly to a low value (0.25), with the very bottom of the diagram representing the contingency in which the precaution power is set randomly to a high value (0.75). With the low precaution power, each party is likely to receive a low signal of precaution power rather than a high signal. Indeed, with the default parameter values, there is for each party approximately an 88\% chance of receiving the signal corresponding to the correct precaution power level. Because the parties' signals are of the true value of precaution power, they will be correlated with one another. The chance probabilities are calculated based on the assumption that a draw from a random distribution is added to the true value and thus becomes noise that obfuscates it. The degree of noise determines the quality of the party's information, and here we assume that each party has equally good information about precaution power.

  \begin{figure}[t]
    \centering
    \includegraphics[width=\textwidth]{../Figures/smalltree_beginning.pdf}
    \caption{Stage I: hidden state, private signals, activity/precaution choice, and accident realization (illustrative tree with two levels for readability).}
    \label{fig:smalltree_beginning.pdf}
  \end{figure}

The plaintiff’s and defendant’s liability signals are private; each party knows the model primitives and distributions but not the other’s realized signal or the hidden state. Actions in the litigation block to be discussed below are observed as they occur (filing, answering, offers, and quits). A party's information set consists of any signal that the party receives, the party's own decisions earlier in the game, and public decisions of the opponent. The court’s internal signal is not observed by the parties; they observe only the binary verdict. Each party has perfect recall, meaning that its own decisions and the signals available to it remain part of its information set throughout the game. The extensive form game diagrams denote information sets by placing a number next to a letter; for example, "D26" refers to a particular information set faced by the defendant at the time the defendant must determine whether to engage in the activity. The algorithm for finding a Nash equilibrium must assign a single mixed strategy (i.e., probability of choosing each action) for each information set, even though an information set may recur many times in the game tree.

Following receipt of signals, the defendant must choose whether to engage in the underlying activity. By engaging in the activity, the defendant receives some benefit. This might be seen as the benefit of driving instead of staying home or pursuing a business activity instead of investing in diversified mutual funds. The value of the benefit parameter is assumed to be $1.0 \times 10^{-3}$ if undertaken. This is a relatively high value relative to some of the other parameters in this simulation, and thus unless otherwise noted, the defendant always chooses to engage in the activity. This approach, however, opens up the possibility of considering not only the possibility that there might be negative consequences from underdeterrence, but also that overdeterrence of activities that have the potential to lead to tort liability could produce costs (or alternatively benefits, if the relevant activities have negative externalities apart from the torts that may be committed, though such externalities are beyond this article's scope). If the defendant decides not to engage in the underlying activity, then the defendant cannot cause injury to the plaintiff or be thought to have been responsible for any injury to the plaintiff, and so the game ends. The modeling cost of including this branch in the model is very low because it does not exponentially increase the size of the tree.

After deciding to engage in the activity, the defendant must choose a level of precaution. The lowest level of precaution is always zero, so given the assumption in this miniaturized model of only two levels of precaution, in this example the choice is the binary one whether to take a precaution or not. In the non-miniaturized game trees providing the article's actual results, the choice can span multiple layers of precaution. The incentive for greater precaution is a reduced likelihood of an accident and thus a stronger position in any hypothetical litigation. The printed probabilities on the accident branches are consistent with increased precaution reducing risk: approximately $1.10 \times 10^{-4}$ at the zero level of precaution is a higher risk than the $8.12 \times 10^{-5}$ at the higher level of precaution. These numbers are calculated via the formulas discussed in detail in the appendices.

These probability values are intentionally quite small. The small values are consistent with an environment in which accidents are highly unlikely to occur but quite costly relative to precaution cost in the event that they do occur. For example, someone driving at a high rate of speed still will generally have only a low probability of engaging in an automobile accident. Likewise, even when a dock owner improperly secures a barge, most of the time it won't drift away from the pier. An advantage of the computational game theory algorithm discussed here is that it easily accommodates this common feature of tort precaution, simply calculating the relevant probability values during its walks through the tree. By contrast, some Monte Carlo computational models might need a relatively high accident rate so that they can generate enough cases in which accidents occur to create a meaningful quantity of data. Informal experimentation, however, suggests that the results in the paper are generally qualitatively similar if accident is made much more common and precaution cost, correspondingly higher.

One detail that the formal extensive form game tree obscures is that the model allows not only for the possibility that the defendant causes an injury to the plaintiff, but also that the defendant is wrongfully concluded to have caused the plaintiff's injury. The probability of a wrongful attribution of liability (conditional on the defendant engaging in the activity but not actually causing injury to the plaintiff) is set in the baseline at $1.0 \times 10^{-5}$. This value is set so that for the values of the parameters that we choose, such wrongful attribution remains less likely than proper attribution of liability. The rare event of wrongful attribution provides a channel through which we can analyze the welfare implications of error and the efficiency of procedural mechanisms. Our analysis will also consider the implications of changing this baseline value.

The accident probabilities listed reflect the sum of the probability that the defendant causes injury to the plaintiff and that the defendant is wrongfully believed to have caused such injury. This nuance is omitted from the game tree because it has no effect whatsoever on game play. Though a more sophisticated model might explicitly model causation, that is not modeled here. Whatever the route to an injury relevant to the model, the legal attribution of causation to the defendant is assumed. The trial itself concerns only whether the defendant exercised standard care, not the issue of causation. Thus, after applying the proper formulas to identify the total accident probability, we can ignore the causation issue in finding perfect Bayesian Nash equilibria. Then, in the final accounting, we can break down the proportion of cases in which causal attribution is wrongful if desired, as will be apparent in diagrams reporting results from simulations. Note that the model does not reflect that the plaintiff may be injured by someone other than the defendant, because that has no bearing on the defendant's liability or on game play. For welfare accounting, in any event, only injuries physically caused by the defendant ($Z=1$) are included; injuries that may occur under wrongful attribution ($Z=0$) are excluded, since the welfare measure captures only the consequences of the defendant’s own activities.

If an accident occurs, the plaintiff decides whether to file suit after observing the plaintiff's private signal. If the plaintiff does file suit, the defendant then decides whether to answer (contest) or to default. A default ends the case immediately with a damages transfer per the liability rule; answering moves the case into bargaining.

  \begin{figure}[t]
    \centering
    \includegraphics[width=\textwidth]{../Figures/smalltree_mid.pdf}
    \caption{Stage II: filing, answer/default, and entry into bargaining (illustrative tree with two levels for readability).}
    \label{fig:smalltree_mid.pdf}
  \end{figure}

After the plaintiff files and the defendant answers, we insert—before any settlement proposals—the parties’ contingent exit choices: the plaintiff may elect to abandon if bargaining later fails, and the defendant may elect to default if bargaining later fails. This placement is deliberate. Modeling these “quit if no settlement” decisions \emph{after} every possible offer profile would require intersecting abandon/default with the settlement subgame, exponentially multiplying information sets without changing what the parties can do or the distribution of outcomes. By recording the quit choices immediately after the file/answer node (and keeping them unobserved until they matter), we preserve most of the strategic richness while dramatically simplifying the tree.

Bargaining then proceeds exactly as in the models of Abramowicz (2025a) and Abramowicz (2025b). The plaintiff states a minimum monetary demand $\mathcal{o}_P$ (labeled “P Offer” in the tree diagram) and the defendant announces the most it is willing to pay $\mathcal(o)_D$ (“D Offer”). Although the diagram prints P Offer before D Offer, this ordering is irrelevant. Because the defendant does not receive information about the plaintiff offer before announcing its own amount, the decisions are effectively simultaneous. If the offers meet or cross (the defendant's offer is at least as great as the plaintiff's), the case settles at the midpoint of the two numbers and litigation ends. The defendant then pays to the plaintiff the amount of the settlement. In addition, each party pays a cost for filing or answering, as well as a cost of bargaining, but then saves the cost of trial. In the baseline condition, the stakes are normalized to $1$ and $c_{file} = c_{answer} = c_{bargain} = c_{trial} = 0.10$.

If the offers do not meet, we implement the contingent choices recorded earlier: if the plaintiff chose to abandon, the case terminates without judgment; if the defendant chose to default, judgment is entered against the defendant with the applicable damages multiplier but without fee-shifting; if both parties would quit, Chance determines which party quits first, thus saving the other the trouble. This scenario, along with successful settlements, is illustrated in Figure \ref{fig:smalltree_end.pdf}

  \begin{figure}[t]
    \centering
    \includegraphics[width=\textwidth]{../Figures/smalltree_end.pdf}
    \caption{Stage III: simultaneous offers and resolution of case in which settlement fails and both parties precommitted to quit).}
    \label{fig:smalltree_end.pdf}
  \end{figure}

If neither party precommitted to quit, the case proceeds to trial. At trial the court observes an imperfect signal of case quality and returns a binary liability verdict; damages and any fee shifting are applied, and remaining stage costs are assessed. Figure \ref{fig:smalltree_end.pdf} depicts this end phase; we keep notation light here to emphasize the structure of the tree. In this case in the miniaturized version of the game, the court always finds liability. In the non-miniaturized version to be reported later, court decisions are heterogeneous at the baseline parameter values, resulting in a probability of liability between $0$ and $1$ even for a particular pair of precaution power and relative precaution levels. 

  \begin{figure}[t]
    \centering
    \includegraphics[width=\textwidth]{../Figures/smalltree_end_adjudication.pdf}
    \caption{Stage III: resolution by trial of case in which settlement fails and neither party precommitted to quit).}
    \label{fig:smalltree_end_adjudication.pdf}
  \end{figure}

\section*{Appendix A: Formal Model}

This appendix collects the mathematical details underlying the accident-probability and adjudication-resolution functions used in the model. The main text emphasizes intuition and strategic structure, while here we provide the explicit functional forms and parameterizations. 
\paragraph{Hidden state and private signals.}
Nature draws a hidden precaution–power state $\theta\in\Theta=\{1,\dots,n_H\}$ uniformly. Each precaution–power level determines the \emph{center} of a liability–strength distribution for the plaintiff, not a deterministic level. We index $\theta$ onto $(0,1)$ by
\[
\varphi(\theta)\;=\;\frac{\theta}{\,n_H+1\,}.
\]
Conditional on $\theta$, the defendant and plaintiff privately observe liability–strength signals
\[
s_D\in\mathcal{S}_D=\{1,\dots,n_{S_D}\},\qquad
s_P\in\mathcal{S}_P=\{1,\dots,n_{S_P}\}.
\]

These are calculated as follows. Let $\Phi$ denote the standard normal cdf. Each party’s signal is constructed by truncated–normal binning, using party–specific noise and grids. For the plaintiff, define $SP_j=\dfrac{j-\tfrac12}{n_{S_P}}$ for $j=1,\dots,n_{S_P}$ and
\[
\Pr(s_P=j\mid \theta)
=
\frac{
\Phi\!\left(\dfrac{SP_j+\tfrac{1}{2n_{S_P}}-\varphi(\theta)}{\sigma_L^{P}}\right)
-
\Phi\!\left(\dfrac{SP_j-\tfrac{1}{2n_{S_P}}-\varphi(\theta)}{\sigma_L^{P}}\right)
}{
\Phi\!\left(\dfrac{1-\varphi(\theta)}{\sigma_L^{P}}\right)
-
\Phi\!\left(\dfrac{-\varphi(\theta)}{\sigma_L^{P}}\right)
}.
\]
For the defendant, define $SD_j=\dfrac{j-\tfrac12}{n_{S_D}}$ analogously and replace $\sigma_L^{P}$ by $\sigma_L^{D}$. Signals are conditionally independent given $\theta$ (and therefore positively correlated unconditionally through the common hidden state). The unconditional distributions over $s_P$ and $s_D$ follow by averaging the above conditional probabilities over the uniform prior on $\theta\in\Theta$.


Each party’s signal is constructed by the same truncated–normal binning, using party–specific noise and grid. For the plaintiff, define $SP_j=\dfrac{j-\tfrac12}{n_{S_P}}$ for $j=1,\dots,n_{S_P}$ and
\[
\Pr(s_P=j\mid \theta)
=
\frac{
\Phi\!\left(\dfrac{SP_j+\tfrac{1}{2n_{S_P}}-\varphi(\theta)}{\sigma_L^{P}}\right)
-
\Phi\!\left(\dfrac{SP_j-\tfrac{1}{2n_{S_P}}-\varphi(\theta)}{\sigma_L^{P}}\right)
}{
\Phi\!\left(\dfrac{1-\varphi(\theta)}{\sigma_L^{P}}\right)
-
\Phi\!\left(\dfrac{-\varphi(\theta)}{\sigma_L^{P}}\right)
}.
\]
For the defendant, define $SD_j=\dfrac{j-\tfrac12}{n_{S_D}}$ analogously and replace $\sigma_L^{P}$ by $\sigma_L^{D}$. Signals are conditionally independent given $\theta$ (and therefore positively correlated unconditionally through the common hidden state). The unconditional distributions over $LS_i$, $s_P$, and $s_D$ follow by averaging the above conditional probabilities over the uniform prior on $\theta\in\Theta$.

\paragraph{Accident probability.}

The defendant then decides whether to engage in the activity, which yields a baseline benefit of $b_0 = 1.0 \times 10^{-3}$ if undertaken, and, if the defendant engages, chooses a relative precaution level $k \in K = \{0,\dots,n_K-1\}$. Note that the lowest level here corresponds to no precaution undertaken at all. We index the relative precaution level $k$ on $[0,1)$ via $\tau(k) \;=\; \frac{k}{n_K }$.

The probability that the defendant causes an accident depends jointly on the hidden precaution–power $\varphi$ and the chosen precaution level $\tau$:
\[
p_{\mathrm{caused}}(\theta,k)
\;=\;
p_{\min}\!\bigl(\varphi(\theta)\bigr)
\;+\;
\bigl(p_{\max}-p_{\min}\!\bigl(\varphi(\theta)\bigr)\bigr)\,
\bigl(1-\tau(k)\bigr)^{\,\eta\!\bigl(\varphi(\theta)\bigr)}.
\]

The function has three key ingredients. First, $p_{\max}$ is the accident probability at zero precaution. Second, $p_{\min}(\varphi)$ sets a floor that declines with $\varphi$: when precaution power is weak, accidents remain more likely even at maximum care; when precaution power is strong, precaution can reduce risk further. Third, the  $\eta(\varphi)$ curvature parameters determine how fast each probability curve bends downward as precaution increases.

This function is illustrated in Figure \ref{fig:precaution.pdf}, given that $n_K = 8$ and given baseline parameter values as follows:

\[
p_{\max}=10^{-4}, \qquad 
p_{\min}^{\varphi=0}=8\times 10^{-5}, \qquad 
p_{\min}^{\varphi=1}=2\times 10^{-5}.
\]

We interpolate linearly in hidden precaution power:
\[
p_{\min}(\varphi(\theta)) = p_{\min}^{\varphi=0} +
\bigl(p_{\min}^{\varphi=1}-p_{\min}^{\varphi=0}\bigr)\,\varphi(\theta).
\]

The level of $\eta(\varphi)$ is not fixed but follows a linear schedule:
\[
\eta(\varphi(\theta)) \;=\; \eta^{\varphi=0} + 
\bigl(\eta^{\varphi=1}-\eta^{\varphi=0}\bigr)\,\varphi(\theta),
\qquad \eta(\varphi) > 1.
\]
For all parameterizations considered, we impose the constraints $p_{\min}(\varphi) \leq p_{\max}$ for every $\varphi$ and $\eta(\varphi) > 1$ for every $\varphi$. These conditions guarantee that accident probability decreases monotonically with precaution and that marginal returns to precaution vanish as $\tau \to 1$.

Each curve corresponds to a level of precaution power, with higher levels of precaution power below lower levels of precaution power, including at the endpoints. The variation in the $\eta$ parameter, however, ensures that higher precaution-power levels make more of a difference for high relative precaution levels. Note that for $\tau(k) = \tfrac{1}{8}$, the highest precaution-power level is bunched with respect to accident probability with precaution-power levels just below it, whereas the accident probability levels are more evenly spread at $\tau(k) = \tfrac{7}{8}$. If $\eta$ were constant, then there would be an equal degree of spreading at each relative precaution level. For baseline parameter values, we set
\[
\eta^{\varphi=0}=3.0, \qquad \eta^{\varphi=1}=1.5.
\]

For purposes of accident probability, only the hidden state $\theta$ and the defendant’s precaution choice $k$ matter. The court’s role and its liability signal are described separately in the trial subsection below.

\paragraph{Accident probability with wrongful attribution.}
In addition to accidents \emph{caused} by the defendant, the model allows for a small probability that an accident is \emph{wrongfully attributed} to the defendant even when no accident is caused by the defendant. Let $\rho\in[0,1]$ denote this wrongful–attribution parameter (interpreted as the conditional probability of legal attribution given that no accident was caused by the defendant). Given the per–state, per–precaution probability of a caused accident $p_{\mathrm{caused}}(\theta,k)$, we define
\[
p_{\mathrm{wrong}}(\theta,k)\;=\;\bigl(1 - p_{\mathrm{caused}}(\theta,k)\bigr)\,\rho,
\qquad
p_{\mathrm{acc}}(\theta,k)\;=\;p_{\mathrm{caused}}(\theta,k)\;+\;p_{\mathrm{wrong}}(\theta,k).
\]
Equivalently,
\[
p_{\mathrm{acc}}(\theta,k)
\;=\;\rho \;+\; \bigl(1-\rho\bigr)\,p_{\mathrm{caused}}(\theta,k),
\]
so that $p_{\mathrm{acc}}(\theta,k)$ is the total probability that an accident is (legally) attributed to the defendant at $(\theta,k)$, combining true causation and wrongful attribution.

Although defendants choose from a discrete grid $k \in K$ with relative levels $\tau(k)=k/(n_K)$, we define $p_{\mathrm{acc}}(\theta,\tau)$ as a smooth function of continuous $\tau\in[0,1]$. The grid values $\tau_k$ are simply the points at which the defendant can act. This continuous extension allows us to compute derivatives of accident probability with respect to precaution, which will be used in the court’s liability test.

\paragraph{Filing, answering, and quitting.}
Following an accident realization, the plaintiff, who has observed $s_P$, chooses whether to \emph{file}, incurring $c_{file}=0.10$ if she does so. If the plaintiff files, the defendant observes $s_D$ and chooses whether to \emph{answer} or \emph{default}. Answering imposes $c_{answer}=0.10$ and moves the case into bargaining. Default ends the case immediately with a plaintiff judgment for normalized damages $\mu$, without any bargaining or trial costs.

If the defendant answers, each party makes a one–time \emph{quit commitment} that is implemented if bargaining fails: the plaintiff may precommit to \emph{abandon} (ending the case with no judgment), and the defendant may precommit to \emph{default} (ending the case with a judgment of $1$ for the plaintiff). If both precommit to quit and no settlement occurs, a symmetric chance node resolves which commitment is implemented; if neither precommits, the case proceeds to trial if bargaining fails. These commitments are made once, at the entry to bargaining, and cannot be revised thereafter.

\paragraph{Bargaining (simultaneous offers).}
Bargaining consists of a single round of \emph{simultaneous} monetary offers on finite grids $\mathcal{O}_P\subset[0,1]$ and $\mathcal{O}_D\subset[0,1]$. The plaintiff demands $\mathcal{o}_P\in\mathcal{O}_P$ and the defendant offers $\mathcal{o}_D\in\mathcal{O}_D$ simultaneously. The per–party bargaining stage cost $c_{bargain}=0.10$ is incurred by any party that participates in this round.

A settlement occurs if and only if the offers meet: $\mathcal{o}_D\ge \mathcal{o}_P$. When settlement occurs, the transfer is the midpoint of the two offers,
\[
T \;=\; \frac{\mathcal{o}_P+\mathcal{o}_D}{2}\in[0,1],
\]
and the case terminates without trial. No damages multiplier or fee–shifting applies to settlements (this model contains no Rule 68 or offer–of–judgment mechanism). If the offers do not meet, the game implements the previously recorded quit commitments (abandon or default); if neither party precommitted, the case proceeds to trial.

\paragraph{Trial.}
When bargaining fails, the court draws a discrete liability–strength signal
\[
s_C \in \mathcal{S}_C=\{1,\dots,n_{S_C}\},
\]
with distribution conditioned on the hidden precaution–power state $\theta$ by the same truncated–normal mechanism used for party signals. Let $SC_j=\dfrac{j-\tfrac12}{n_{S_C}}$ and let $\sigma_L^{C}>0$ denote the court’s noise parameter. Then
\[
\Pr(s_C=j\mid \theta)
=
\frac{
\Phi\!\left(\dfrac{SC_j+\tfrac{1}{2n_{S_C}}-\varphi(\theta)}{\sigma_L^{C}}\right)
-
\Phi\!\left(\dfrac{SC_j-\tfrac{1}{2n_{S_C}}-\varphi(\theta)}{\sigma_L^{C}}\right)
}{
\Phi\!\left(\dfrac{1-\varphi(\theta)}{\sigma_L^{C}}\right)
-
\Phi\!\left(\dfrac{-\varphi(\theta)}{\sigma_L^{C}}\right)
},
\qquad j=1,\dots,n_{S_C}.
\]
Conditional on $\theta$, the court signal is independent of the parties’ signals; the correlation among all signals arises through the shared hidden state. In the baseline model, the court does not condition its liability determination on whether an accident has occurred. This reflects the idea that the Hand-formula style negligence test is applied ex ante, using only cost and risk information rather than case-selection data. As a result, the court’s decision depends solely on its noisy observation of $\theta$, not on the fact that the dispute reached litigation. Formally, we model
\[
\Pr(V=1 \mid \theta,k,s_C) \quad \text{rather than} \quad \Pr(V=1 \mid \theta,k,s_C,\text{accident}),
\]
so that the court’s liability probability is unaffected by the event that a case is filed. This independence from case selection applies only to the court’s negligence test itself. In Appendix B, ex-ante probabilities are updated conditional on whether an accident was realized, but this conditioning affects only Bayesian accounting of states and signals. It does not alter the court’s liability rule $\ell(c,k)$, which is applied solely to cost and risk information as if ex ante.

Fix a precaution level $k\in K=\{0,\dots,n_K-1\}$ and let $\tau_k=\tau(k)$. For each hidden state $\theta$, recall that the accident probability is
\[
p_{\mathrm{acc}}(\theta,\tau)
=\rho+(1-\rho)\Bigl[p_{\min}\!\bigl(\varphi(\theta)\bigr)
+\bigl(p_{\max}-p_{\min}\!\bigl(\varphi(\theta)\bigr)\bigr)\,(1-\tau)^{\eta(\varphi(\theta))}\Bigr].
\]
The marginal reduction in accident risk at $\tau_k$ is defined by the continuous derivative taken with a positive sign:
\[
\Delta(\theta,\tau_k)
=(1-\rho)\,\Bigl(p_{\max}-p_{\min}\!\bigl(\varphi(\theta)\bigr)\Bigr)\,\eta\!\bigl(\varphi(\theta)\bigr)\,(1-\tau_k)^{\eta(\varphi(\theta))-1},
\]
so $\Delta(\theta,\tau_k)\ge 0$ for $\tau_k\in[0,1)$ and $\Delta(\theta,1)=0$ (since $\eta(\varphi)>1$). This sign convention interprets $\Delta$ as the benefit of an infinitesimal increase in precaution. We evaluate this marginal benefit with respect to the legally attributed accident probability rather than the physical probability of causation. The court is modeled as aware that some attributions are wrongful in expectation, but unable to distinguish them case by case. Liability therefore attaches to all attributed accidents, and the marginal-benefit measure is discounted by $(1-\rho)$. This approach plausibly reflects the institutional perspective of a court that applies the Hand formula to the risk of being held liable, not to the risk of true causation.

Given a realized court signal $s_C=c$, Bayes’ rule gives the posterior distribution $\Pr(\theta \mid s_C=c)$. The expected marginal reduction at $(c,k)$ is
\[
\bar{\Delta}(c,k)=\sum_{\theta\in\Theta}\Delta(\theta,\tau_k)\,\Pr(\theta\mid s_C=c).
\]

Because accident harm is normalized to $1$, the expected benefit of additional precaution equals $\bar{\Delta}(c,k)$. Let $c_{\mathrm{prec}}>0$ denote the cost of one discrete step in $k$; because $\tau(k)=k/n_K$, the cost per unit of the continuous precaution index $\tau$ is $n_K\,c_{\mathrm{prec}}$. The court forms the ratio
\[
R(c,k)=\frac{\bar{\Delta}(c,k)}{\,n_K\,c_{\mathrm{prec}}\,}.
\]
Here $\bar{\Delta}(c,k)$ is measured in accident probability points per unit of the continuous precaution index $\tau$, so dividing by $n_K\,c_{\mathrm{prec}}$ (the cost per unit $\tau$) yields a dimensionless ratio.

For a negligence threshold parameter $\lambda \ge 0$, the verdict is
\[
V=\mathbf{1}\!\left\{\,R(c,k)>\lambda\,\right\}.
\]
The baseline analysis sets $\lambda=1$, corresponding to the familiar benefit–cost test. We also examine alternative values $\lambda\in\{0,0.8,1.2,2\}$ to capture more lenient or stricter liability standards. 

Although defendants choose from the discrete grid $k\in\{0,\dots,n_K-1\}$ with $\tau_k=k/n_K$, the court applies a continuous marginal test. This means there is no discrete safe harbor at the top step $k=n_K-1$: the court evaluates the benefit of further precaution via $\Delta(\theta,\tau_{n_K-1})>0$ whenever $\tau_{n_K-1}<1$. This reflects the modeling philosophy that real-world care is refinable even when our action set is discretized for computation. As $n_K$ increases, the discrete grid approximates the continuous ideal more closely and the probability mass on “negligent at the top step” correspondingly shrinks. Parameterizations that imply frequent liability at high $k$ are therefore interpretable as environments with unusually high residual returns to care (large $\eta(\varphi)$ or wide $p_{\max}-p_{\min}(\varphi)$) or relatively low marginal cost (small $c_{\mathrm{prec}}$), not as artifacts of the liability test.


The probability of liability at trial, conditional on the hidden state $\theta$ and the chosen precaution level $k$, is
\[
\Pr(V=1\mid \theta,k)\;=\;\sum_{c\in\mathcal{S}_C}\mathbf{1}\!\left\{R(c,k)>\lambda\right\}\,\Pr(s_C=c\mid \theta),
\qquad
\Pr(V=0\mid \theta,k)=1-\Pr(V=1\mid \theta,k).
\]
These probabilities appear explicitly in the game tree as branches following the court’s signal draw.

\paragraph{Final wealth across all case outcomes.}
Initial wealths are $W_P^{(0)}=W_D^{(0)}=10$. We also introduce a binary cause flag $Z\in\{0,1\}$, where $Z=1$ indicates that the defendant physically caused the accident and $Z=0$ indicates that any legally attributed accident was not physically caused by the defendant (i.e., wrongful attribution). Let $C_P$ and $C_D$ denote the cumulative stage costs actually incurred along the realized path, where the baseline stage costs are
\[
c_{file}=c_{answer}=c_{bargain}=c_{trial}=0.10.
\]
The defendant’s decision to engage in the activity yields a baseline benefit $b_0>0$. If the defendant chooses not to engage, social welfare is assigned a negative baseline of $-b_0$. This convention ensures that both the cost of precaution and the opportunity cost of forgoing the activity enter social welfare with the same sign, so that precaution expenditures and non-engagement are treated consistently as resource costs offsetting the benefits of accident reduction.

\emph{Settlement.} If the parties reach settlement, with offers $\mathcal{o}_P$ and $\mathcal{o}_D$ meeting, the transfer is
\[
T \;=\; \frac{\mathcal{o}_P+\mathcal{o}_D}{2}.
\]
Final wealths are
\[
W_P^{\mathrm{final}} \;=\; W_P^{(0)} - C_P + T,
\qquad
W_D^{\mathrm{final}} \;=\; W_D^{(0)} - C_D - T + b_0.
\]
No damages multiplier or fee–shifting applies to settlements.

\emph{Trial.} If a trial occurs, let $V\in\{0,1\}$ denote the court’s verdict for the plaintiff and $J \;=\; \mu \cdot V$ the judgment, where $\mu\ge 0$ is the damages multiplier. Each side pays $c_{trial}$ in addition to earlier costs. Fee–shifting is represented by $\phi\in[0,1]$, the fraction of the prevailing party’s \emph{total} costs reimbursed by the loser (American rule: $\phi=0$; full loser–pays: $\phi=1$). Final wealths under trial are
\[
W_P^{\mathrm{final}}
\;=\;
W_P^{(0)} - C_P + J + \phi\bigl(V\,C_P - (1-V)\,C_D\bigr),
\]
\[
W_D^{\mathrm{final}}
\;=\;
W_D^{(0)} - C_D - J + \phi\bigl((1-V)\,C_D - V\,C_P\bigr) + b_0.
\]
In the baseline with $\mu=1$ and $\phi=0$, these reduce to $W_P^{(0)}-C_P+V$ and $W_D^{(0)}-C_D-V+b_0$.

\emph{Default.} If the defendant defaults at the answering stage, the plaintiff receives a judgment of $\mu$, reflecting the damages multiplier, without incurring $c_{bargain}$ or $c_{trial}$. Fee–shifting does not apply.
Final wealths are
\[
W_P^{\mathrm{final}} \;=\; W_P^{(0)} - C_P + \mu,
\qquad
W_D^{\mathrm{final}} \;=\; W_D^{(0)} - C_D - \mu + b_0,
\]

where here $C_P=c_{file}$ and $C_D=0$ along this path. If default is implemented via a quit commitment \emph{after} bargaining fails, the same expressions apply with $C_P$ and $C_D$ including $c_{bargain}$ (and earlier costs) but not $c_{trial}$.

\emph{Abandon.} If the plaintiff abandons through a quit commitment, no judgment is entered. Neither damages multipliers nor fee–shifting apply. Final wealths are
\[
W_P^{\mathrm{final}} \;=\; W_P^{(0)} - C_P,
\qquad
W_D^{\mathrm{final}} \;=\; W_D^{(0)} - C_D + b_0,
\]
where $C_P$ and $C_D$ include all stage costs already incurred along the realized path.

\emph{Defendant quit.} If the defendant defaults through a quit commitment after bargaining fails, the plaintiff receives a judgment of $\mu$, reflecting the damages multiplier, but fee–shifting does not apply. Final wealths are
\[
W_P^{\mathrm{final}} \;=\; W_P^{(0)} - C_P + \mu,
\qquad
W_D^{\mathrm{final}} \;=\; W_D^{(0)} - C_D - \mu + b_0,
\]
with $C_P$ and $C_D$ reflecting costs incurred through the bargaining stage (and earlier), but not $c_{trial}$ since no trial occurs.

\paragraph{Final utilities.}
Let $W_i^{\mathrm{final}}$ denote party $i$'s final wealth after the litigation path has concluded.

\emph{Risk–neutral case.} With risk neutrality, utilities coincide with final wealth:
\[
U_i \;=\; W_i^{\mathrm{final}}.
\]

\emph{Risk–averse cases.} With risk aversion, we adopt a constant–absolute–risk–aversion specification,
\[
U_i \;=\; -\,\exp\!\bigl(-\alpha\,W_i^{\mathrm{final}}\bigr),
\]
where $\alpha\ge 0$ is the risk–aversion coefficient. The four calibrations used are:
\[
\alpha = 0 \quad\text{(none / risk neutral)},\qquad
\alpha = 1 \quad\text{(mild)},\qquad
\alpha = 2 \quad\text{(moderate)},\qquad
\alpha = 4 \quad\text{(high)}.
\]
With baseline wealths of $10$ and stakes of size $1$, these values span from linear preferences to substantial curvature in the utility of wealth.

\emph{Welfare reporting.} Aggregate social welfare is defined as the negative of the real resource costs:
\[
Welfare \;=\; -\bigl(\,k \cdot c_{\mathrm{prec}} \;+\; \mathbf{1}_{\{Z=1\}} \;+\; (C_P+C_D)\,\bigr)
\;-\;\mathbf{1}_{\{\neg \text{engage}\}}\,b_0.
\]
Here $c_{\mathrm{prec}}$ denotes the cost of taking one discrete step in the precaution index $k$, so that $k \cdot c_{\mathrm{prec}}$ is the total precaution expenditure chosen by the defendant. The indicator $\mathbf{1}_{\{Z=1\}}$ represents the injury loss (equal to $1$ if the accident is physically caused by the defendant, $0$ otherwise). Accidents that occur in wrongful attribution cases ($Z=0$) are not counted, since welfare includes only consequences of the defendant’s own activity. The quantity $C_P+C_D$ is the total litigation cost (the sum of stage costs incurred by both parties). The additional term $\mathbf{1}_{\{\neg \text{engage}\}}\,b_0$ reflects the opportunity cost of forgoing the activity: when the defendant does not engage, welfare is reduced by $b_0$. Monetary transfers between plaintiff and defendant cancel in this welfare measure. This welfare measure is consistent with the earlier sign convention: both precaution expenditures and the opportunity cost of non-engagement are treated as negative contributions to welfare, so that the court’s marginal-benefit test and the welfare accounting operate on the same footing in terms of opportunity costs.




\section*{Appendix B: Bayesian Marginalization of Chance Nodes}

\subsection*{B.1 Motivation and Overview}

Explicitly modeling every chance node (random event) can cause an explosion in the size of the game tree. We thus use Bayesian marginalization of chance nodes to avoid explicitly branching on a random variable at the outset of the game. Instead, we integrate out uncertainties concerning the variable represented by the node on the fly, adjusting probabilities and outcomes accordingly. This approach keeps the tree smaller while still accounting for the uncertainty. In our litigation game, the first chance event determines the precaution power state: the hidden state of which the defendant, plaintiff, and court may eventually receive signals. Instead of selecting such a state, we can convert the game into a mathematically equivalent game. The strategic players' information sets in this game are exactly the same as those in the original game, because the hidden state never becomes part of the plaintiff's or defendant's information set. While in the full game the distribution of the defendant's private signal is calculated conditional on the actual hidden state, in the collapsed tree the distribution of the defendant's private signal is calculated by integrating over all such possible hidden states. Later distributions then require similar treatment. With this approach, any equilibrium of the original extensive form game tree is equivalently an equilibrium of the collapsed game, and vice versa.

Figure \ref{fig:smalltree_collapsed_beginning.pdf} illustrates the beginning of the extensive form game in which the node selecting precaution power is omitted. The first node now represents the defendant's signal. Because this is not conditioned on the hidden state, the defendant is equally likely to receive a low and a high signal. Instead of immediately calculating the plaintiff's signal, the defendant makes its decisions on whether to engage in the activity and if so, what relative level of precaution to take. The next node then represents the accident, and the accident probabilities are calculated by integrating over both possible hidden states in this miniaturized game. Finally, the plaintiff's signal node now follows the accident node. The probability that the plaintiff will receive each of the two possible signals takes into account the defendant's signal, the defendant's precaution level, and the occurrence of the accident, integrating again over both possible hidden states. Note that in this case, the plaintiff is much more likely to receive the low signal, because that is the signal that the defendant received. It would equivalently be possible to position the plaintiff's signal node immediately after the defendant signal and then condition the accident probability on the parties' joint signals, but this approach produces a smaller game tree.

  \begin{figure}[t]
    \centering
    \includegraphics[width=\textwidth]{../Figures/smalltree_collapsed_beginning.pdf}
    \caption{Beginning of the collapsed game.}
    \label{fig:smalltree_collapsed_beginning.pdf}
  \end{figure}

This section explains more formally how those probability adjustments are computed using Bayes’ rule, and how we handle missing (latent) variables by splitting game records with appropriate weights after the play-through. The goal is to produce the same expected outcomes as the full model, but with far fewer nodes in the game tree.

\subsection*{B.2 Bayesian Updates for Chance Probabilities in the Simplified Tree}

When chance nodes are marginalized, we must update the probabilities of subsequent events based on whatever partial information is revealed. We apply Bayes’ rule to revise distributions as new evidence (signals or outcomes) arrives. For each eliminated chance node, we provide the corresponding marginalization formula. The order of presentation follows the order in which these chance nodes would be encountered in play with marginalization: the defendant’s signal, the accident outcome, the plaintiff’s signal, and finally the court’s liability decision.

\paragraph{Defendant’s liability signal and posterior over the hidden state.} 
Let $\Theta=\{1,\dots,n_H\}$ index the hidden precaution–power state and let $s_D\in\mathcal{S}_D=\{1,\dots,n_{S_D}\}$ denote the defendant’s liability signal, with prior $\Pr(\theta)=1/n_H$ and conditional likelihoods $\Pr(s_D=j\mid \theta)$ as in Appendix~A. In the collapsed-tree implementation, we do not draw $\theta$ explicitly before the defendant’s signal is realized. Instead, the chance node for $s_D$ uses the unconditional (marginal) distribution \[ \Pr(s_D=j) \;=\; \sum_{\theta\in\Theta}\Pr(s_D=j\mid \theta)\Pr(\theta) \;=\; \frac{1}{n_H}\sum_{\theta=1}^{n_H}\Pr(s_D=j\mid \theta), \qquad j=1,\dots,n_{S_D}. \] Upon observing a particular realization $d\in\mathcal{S}_D$, beliefs over $\theta$ are updated by Bayes’ rule: \[ \Pr(\theta\mid s_D=d) \;=\; \frac{\Pr(s_D=d\mid \theta)\Pr(\theta)}{\sum_{\theta'\in\Theta}\Pr(s_D=d\mid \theta')\Pr(\theta')} \;=\; \frac{\Pr(s_D=d\mid \theta)}{\sum_{\theta'=1}^{n_H}\Pr(s_D=d\mid \theta')}, \qquad \theta=1,\dots,n_H, \] where the second equality uses the uniform prior $\Pr(\theta)=1/n_H$. The vector $\bigl(\Pr(\theta\mid s_D=d)\bigr)_{\theta\in\Theta}$ serves as the posterior over hidden states that conditions all downstream collapsed-node probabilities (e.g., accident occurrence and court outcomes), while the marginal $\bigl(\Pr(s_D=j)\bigr)_{j\in\mathcal{S}_D}$ determines the branching probabilities at the defendant-signal chance node.

\paragraph{Accident occurrence (Bayesian marginalization).}
Let $A\in\{0,1\}$ indicate whether an accident is legally attributed to the defendant ($A=1$) or not ($A=0$), conditional on engagement in the activity and a fixed precaution level $k\in K$. Appendix~A defined the overall accident probability
\[
p_{\mathrm{acc}}(\theta,k) \;=\; \Pr(A=1 \mid \theta,k),
\]
which aggregates both true causation and wrongful attribution. In the collapsed tree, the hidden state $\theta\in\Theta=\{1,\dots,n_H\}$ is not drawn before the accident node. Instead, after observing the defendant’s signal $s_D=d$, we branch at the accident node using the mixture of state–contingent accident probabilities,
\[
\Pr(A=1 \mid s_D=d,\,k) \;=\; \sum_{\theta\in\Theta} p_{\mathrm{acc}}(\theta,k)\,\Pr(\theta \mid s_D=d),
\qquad
\Pr(A=0 \mid s_D=d,\,k) \;=\; 1-\Pr(A=1 \mid s_D=d,\,k).
\]
If no informative signal has been observed, the uniform prior $\Pr(\theta)=1/n_H$ is used in place of $\Pr(\theta \mid s_D=d)$.

\medskip
\noindent\textbf{Belief updating after the accident realization.}
When the outcome $A$ is realized, beliefs over $\theta$ are refined by Bayes’ rule. For $A=1$,
\[
\Pr(\theta \mid s_D=d,\,A=1,\,k)
\;=\;
\frac{p_{\mathrm{acc}}(\theta,k)\,\Pr(\theta \mid s_D=d)}
{\sum_{\theta'\in\Theta} p_{\mathrm{acc}}(\theta',k)\,\Pr(\theta' \mid s_D=d)}\,,
\qquad \theta\in\Theta,
\]
and for $A=0$,
\[
\Pr(\theta \mid s_D=d,\,A=0,\,k)
\;=\;
\frac{\bigl(1-p_{\mathrm{acc}}(\theta,k)\bigr)\,\Pr(\theta \mid s_D=d)}
{\sum_{\theta'\in\Theta} \bigl(1-p_{\mathrm{acc}}(\theta',k)\bigr)\,\Pr(\theta' \mid s_D=d)}\,,
\qquad \theta\in\Theta.
\]
These posteriors become the belief states conditioning all downstream collapsed chance nodes, in particular the plaintiff’s signal.

\paragraph{Plaintiff’s liability signal.}
The plaintiff’s liability signal is drawn only after both the defendant’s signal and the accident outcome have been realized. Its distribution thus depends on three factors: the defendant’s observed signal $s_D=d$, the relative precaution level $k$ chosen earlier in the game, and the accident outcome $A\in\{0,1\}$. Each of these observations filters the set of possible hidden states. The defendant’s signal provides an initial posterior, the occurrence or non-occurrence of an accident at precaution level $k$ further refines that posterior, and the precaution level itself matters because it shapes the accident probabilities $p_{\mathrm{acc}}(\theta,k)$ that enter those Bayesian updates.

Formally, the branching probabilities for the plaintiff’s signal are
\[
\Pr(s_P=j \mid s_D=d,\,A,\,k)
\;=\;\sum_{\theta\in\Theta} \Pr(s_P=j \mid \theta)\,\Pr(\theta \mid s_D=d,\,A,\,k),
\qquad j=1,\dots,n_{S_P}.
\]
Here $\Pr(\theta \mid s_D=d,\,A,\,k)$ is the posterior over hidden states derived in the accident step. Thus, the plaintiff’s evidence is generated from a mixture of state–conditional signal distributions weighted by beliefs that already incorporate the defendant’s signal, the accident outcome, and the level of precaution taken.

Observing a specific plaintiff signal $s_P=p$ then produces an additional Bayesian update:
\[
\Pr(\theta \mid s_D=d,\,A,\,k,\,s_P=p)
\;=\;
\frac{\Pr(s_P=p \mid \theta)\,\Pr(\theta \mid s_D=d,\,A,\,k)}
{\sum_{\theta'\in\Theta} \Pr(s_P=p \mid \theta')\,\Pr(\theta' \mid s_D=d,\,A,\,k)},
\qquad \theta\in\Theta.
\]
In other words, the plaintiff’s signal functions as a further noisy measurement of the latent precaution–power state, layered on top of the information already provided by the defendant’s signal, the accident outcome, and the chosen precaution level. The resulting posterior conditions the final stage of the game, the court’s liability decision.

\paragraph{Court’s liability decision.} The final chance event in the collapsed tree is the court’s verdict at trial. Let $V\in\{0,1\}$ denote the court’s binary decision, with $V=1$ indicating a finding of liability. In the explicit game tree of Appendix~A, the verdict depends on the hidden state $\theta$, the chosen precaution level $k$, and the court’s own noisy signal $s_C\in\mathcal{S}_C$ of precaution power. Because the hidden state is marginalized in the collapsed tree, we branch at the court node using the posterior beliefs about $\theta$ formed from all previous observations.

For each hidden state $\theta$, Appendix~A defined the conditional law $\Pr(s_C=c\mid \theta)$, $c\in\mathcal{S}_C$. In the collapsed tree, once the defendant’s signal $s_D$, accident outcome $A$, precaution level $k$, and possibly the plaintiff’s signal $s_P$ have been observed, the model holds a posterior distribution $\Pr(\theta \mid s_D, A, k, s_P)$ over hidden states. The unconditional distribution of the court’s signal is therefore
\[
\Pr(s_C=c \mid s_D, A, k, s_P)
= \sum_{\theta\in\Theta} \Pr(s_C=c\mid \theta)\,\Pr(\theta \mid s_D, A, k, s_P).
\]

For each $(c,k)$ pair, the liability rule determines whether the court would declare the defendant liable. Let $\ell(c,k)\in\{0,1\}$ be the indicator that the benefit–cost ratio of additional precaution exceeds the negligence threshold $\lambda$ given court signal $c$ at precaution level $k$ (as specified in Appendix~A). The probability of a liability verdict in the collapsed tree is then
\[
\Pr(V=1 \mid s_D, A, k, s_P)
= \sum_{c\in\mathcal{S}_C} \ell(c,k)\,
   \Pr(s_C=c \mid s_D, A, k, s_P),
\qquad
\Pr(V=0 \mid \cdot)=1-\Pr(V=1 \mid \cdot).
\]

If the game reaches trial and a verdict $V$ is realized, beliefs over the hidden state are updated by Bayes’ rule. For a liability verdict,
\[
\Pr(\theta \mid s_D, A, k, s_P, V=1)
= \frac{\Bigl(\sum_{c:\,\ell(c,k)=1}\Pr(s_C=c\mid \theta)\Bigr)\,
       \Pr(\theta \mid s_D, A, k, s_P)}
       {\sum_{\theta'\in\Theta}\Bigl(\sum_{c:\,\ell(c,k)=1}\Pr(s_C=c\mid \theta')\Bigr)\,
        \Pr(\theta' \mid s_D, A, k, s_P)},
\]
and for a no-liability verdict,
\[
\Pr(\theta \mid s_D, A, k, s_P, V=0)
= \frac{\Bigl(\sum_{c:\,\ell(c,k)=0}\Pr(s_C=c\mid \theta)\Bigr)\,
       \Pr(\theta \mid s_D, A, k, s_P)}
       {\sum_{\theta'\in\Theta}\Bigl(\sum_{c:\,\ell(c,k)=0}\Pr(s_C=c\mid \theta')\Bigr)\,
        \Pr(\theta' \mid s_D, A, k, s_P)}.
\]
These posteriors provide the probability weights over hidden states that are used to assign concrete values when outcomes are tabulated.

\subsection*{B.3 Ex Post Splitting of Game Records (Matches Implementation)}

We record realized play and then, when needed for reporting, clone records to fill in the hidden state $\theta$ and whether a legally attributed accident is truly caused by the defendant. The cloning weights are chosen to reproduce the expectations of the explicit tree.

\paragraph{Latent pieces.}
Let $\theta\in\Theta=\{1,\dots,n_H\}$ be the hidden precaution--power state. Introduce a cause flag $Z\in\{0,1\}$ with $Z=1$ if the accident is \emph{truly caused by the defendant} and $Z=0$ otherwise. Let $s_D=d$ and $s_P=p$ be the realized defendant and plaintiff liability signals, and let $k$ be the realized precaution level. Denote by
\[
\pi_\theta \;=\; \Pr\!\bigl(\theta \mid s_D=d,\,\text{accident flag},\,k,\,s_P=p,\,\text{verdict (if any)}\bigr)
\]
the posterior over hidden states implied by the collapsed updates (B.2). Let
\[
q_{\mathrm{wrong}}(d,p,k)\in[0,1]
\]
be the probability of \emph{wrongful attribution} given the signals and precaution level; this is the same quantity computed in the code path via the risk model’s signal-based function.

\paragraph{Split rule.}
Create one clone for each $(\theta,z)\in\Theta\times\{0,1\}$ with weights $w(\theta,z)$:

\emph{No accident occurs.} There is nothing to split:
\[
w(\theta,1)=0,\qquad w(\theta,0)=\pi_\theta,\qquad \theta\in\Theta.
\]

\emph{An accident occurs.} Split by the signal-based wrongful-attribution probability:
\begin{align*}
w(\theta,1) &= \pi_\theta\,\bigl(1 - q_{\mathrm{wrong}}(d,p,k)\bigr),\\
w(\theta,0) &= \pi_\theta\, q_{\mathrm{wrong}}(d,p,k), \qquad \theta\in\Theta.
\end{align*}

\paragraph{Averaging outcomes.}
For any statistic $X$ that depends on $(\theta,Z)$ through the primitives,
\[
\mathbb{E}\!\left[X \mid s_D=d,\,\text{accident flag},\,k,\,s_P=p,\,\text{verdict}\right]
=\sum_{\theta\in\Theta}\sum_{z\in\{0,1\}} w(\theta,z)\,X(\theta,z).
\]

\paragraph{Plaintiff signal not yet observed and no accident.}
If $s_P$ was not realized (no accident branch) and reporting requires splitting by $\theta$, sum first over possible $p$ using the mixture $P(s_P=p\mid s_D=d,\text{no-accident},k)$ and then apply the weights above (which, without an accident, reduce to $w(\theta,0)=\pi_\theta$).



\subsection*{B.4 Validation by Game-Tree Equivalence}

We verified the correctness of Bayesian marginalization by direct computational comparison. Two game trees are constructed: the original, with explicit chance nodes for hidden states and signals, and the collapsed version, with Bayesian marginalization. Both trees contain the same information sets $\mathcal{I}$, so that a random number generator $r:\mathcal{I}\to(0,1)$ can be used to generate random mixed strategies by assigning to each information set a probability distribution with all entries positive. Specifically, for each $I\in \mathcal{I}$ and each available action $a\in A(I)$, we set
\[
\sigma(a\mid I) \;\propto\; r(I,a),
\]
and normalize to obtain a probability distribution $\sigma(\cdot\mid I)$. This ensures that each information set is assigned a nondegenerate randomized strategy.

Given the same realization of $r$, we then compute players’ expected utilities in both the original and collapsed games. Because the marginalization preserves the underlying probability law, the expected utilities coincide up to floating-point precision. In our implementation, the discrepancy is bounded by $10^{-12}$. Such rounding differences can be magnified over many iterations of regret-based algorithms, but they have no bearing on whether the outcome of a solution process constitutes an approximate Bayesian equilibrium.



\printbibliography
\end{document}
